\chapter{Introduction}

Multi-robot search is an important problem due to emerging application domains such as target tracking \cite{schlotfeldt2018anytime}, search and rescue \cite{jennings1997cooperative}, and industrial inspection \cite{correll2009multirobot}.
However, the multi-robot search problem remains unsolved, mainly due to three NP-hard challenges: the maximal coverage problem, multi-robot task allocation, and minimal routing problem.
Challenges of these problems are as follows.
First, finding the maximal coverage under a budget is computational intractability \cite{khuller1999budgeted}.
Second, the task allocation to multiple robots is another issue \cite{korsah2013comprehensive}.
Third, each robot in multi-robot search problems needs to solve the traveling salesman problem (TSP) \cite{zhang2016submodular}.

Recent approaches to solving these problems are as follows.
For the multi-robot search problem,
the researchers propose an efficient path planning algorithm (eMIP) \cite{singh2007efficient}.
The problem is solved by a sequential single robot path planning algorithm.
Furthermore, each robot needs to solve the TSP, which results in poor time complexity.
A theoretical performance guarantee is provided.
However, the search performance deteriorates as the number of robots increases.
The problem is reformulated as submodular maximization subject to intersection system constraints (MRSIS) in \cite{li2024mrsis}.
The search problem is solved by generating a set of robot trajectories while considering the balance of task assignments. However, its theoretical performance is not provided.

For multi-robot task allocation problems, Markov Decision Processes (MDP) over graphs are adopted.
In \cite{paull2022learning}, the researchers propose a reinforcement learning algorithm based on a graph neural network.
The task assignment problem is solved by selecting the node with the highest probability for each robot.
However, the routing problem and workload balance are not considered.

\begin{figure}[htbp]
\centerline{\includegraphics[width=1\textwidth]{overview-3.png}}
\caption{Overview of the proposed multi-robot search system with three robots.
(a) An example of subgoals (nodes) in a search environment.
(b) The proposed algorithm (MRSM) finds a solution $S$ that maximizes an objective function $F$ subject to resource constraints $\mathcal{M}$.
(c) The circular sectors are the coverage areas and the color arrows (orange, blue, and green) are robot trajectories. The robots follow trajectories generated by the MRSM and find targets in the environment.
}
\label{overview}
\end{figure}


%\begin{figure}[htbp]
%\centering
%\includegraphics[width=1.0\linewidth]{method_intro.jpg}
%\caption{ Illustration of the proposed method.
%(a) Subgoals.
%The blue points and decimal numbers represent subgoals and the index of subgoals, respectively.
%(b) The cost-benefit spanning tree.
%The green points, red lines and decimal numbers represent nodes in spanning tree, edges in spanning tree and the index of subgoals, respectively.
%(c) Path.
%The green points and red lines represent the path nodes and path edges, respectively.
%}
%%(b) The change of objective function from minimum spanning tree to cost-benefit spanning tree (c) Tree structure of \emph{CBST}.}
%\label{fig:method_intro}
% \end{figure}


For search via learning approaches,
the researchers propose a probability density function (PDF) as a reward function that generates a sequence of decisions based on the reinforcement learning method \cite{sheng2022pd}.
The search problem and the allocation problem are solved by selecting actions with the maximal individual reward function for each robot.
Yet, the graph only considers the unit cost for routing, and the balance of the robots' workloads is not taken into account, potentially resulting in poor task assignment.

To resolve the aforementioned issues, this research proposes a Multi-Robot Search with Matroid constraints (MRSM) algorithm.
The key concept of the MRSM is to simultaneously solve task allocation, minimal routing, and maximal coverage problems.
Maximal coverage and balanced workloads are considered simultaneously.
The coverage function is to cover a larger area for robots, while the balancing function is to measure the equality of the workloads assigned to robots.
Besides, the routing constraint is reformulated as matroid to boost the theoretical guarantees.

% explain my approaches
The proposed MRSM method is illustrated in Fig. \ref{overview}. Given an environment map with subgoals, the goal is to find all targets with multiple robots as soon as possible.
In Fig. \ref{overview}(a), subgoals are evenly distributed in the space. A weighted graph $\mathcal{G}(V,E, w)$ is constructed, where $V$ represents the subgoal set, $E$ represents the edge set, and $w$ is an Euclidean distance function.
In Fig. \ref{overview}(b), the MRSM generates a set of trajectories that maximize the environment coverage and maintain balanced workloads among robots.
In Fig. \ref{overview}(c), the robots visit the subgoals and search for targets in the environment.

% My contributions
The contributions of this research are as follows.
First,  the submodular maximization under matroid intersection constraints (MRSM) is proposed for multi-robot search problems.
To the best of our knowledge, this is the first work to propose this objective for these problems.
Second, thanks to the submodularity,
the theoretical guarantees of MRSIS \cite{li2024mrsis} and MRSM are proved as $\frac{1}{2+k_G} \overline{OPT}$ and $\frac{1}{3}\widetilde{OPT}$, respectively, where $k_G$ is the ratio of the maximum to the minimum cardinality ($k_G > 1$), $\overline{OPT}$ is an approximately optimal solution under a graph structure and $\widetilde{OPT}$ is an approximately optimal solution under a spanning tree structure.\footnote{A spanning tree is a tree structure that can be derived from an undirected graph and includes all of the graph's vertices. A minimum spanning tree is a spanning tree whose sum of edge weights is as small as possible.}
To the best of our knowledge, the constant factor of MRSM theoretical bound is better than the that of prior work \cite{li2024mrsis}.
Notice that the theoretical bounds that of MRSIS \cite{li2024mrsis} and eMIP \cite{singh2007efficient} depend on variables.
The key novelty of the proposed method (MRSM), compared to recent approaches, is that MRSM considers clustering and routing budgets within the matroid constraint and achieves a constant theoretical bound.
Third, simulated and real-world experiments demonstrate that the MRSM outperforms the benchmark algorithms (e.g., MRSIS \cite{li2024mrsis}, CapAM \cite{paull2022learning} and PD-FAC \cite{sheng2022pd}). Furthermore, to clarify the parameter influence, a comprehensive analysis is presented for deployment in specific applications.
The experiment results show that MRSM outperforms state-of-the-art approaches (e.g., MRSIS \cite{li2024mrsis}, CapAM \cite{paull2022learning} and PD-FAC \cite{sheng2022pd}) regarding expected time to detection (ETTD) in the multi-robot search problem.

In this research, some assumptions are made. First, MRSM relies on known environments. A set of subgoals is evenly distributed in the search environments.
Second, the coverage at every subgoal can be pre-computed since the search method is based on known maps.
Third, the perception of robots includes uncertainty and targets may be occluded in the environment.

% Organization of the paper
This paper is organized as follows. Section 2 reviews the relevant work on target search methods, multi-robot task allocation, and routing constraints. Section 3 describes the background knowledge of this research. Section 4 introduces the problem formulation. Section 5 describes the search algorithm. Section 6 describes the experiments and analyzes the results. Finally, Section 7 draws conclusions and outlines future work.

\section{Publication Note}
Portions of the literature survey, problem formulation, and experiments appeared in \cite{li2024mrsis} \cite{li2024casmo}.

