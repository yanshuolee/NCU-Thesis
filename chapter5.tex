\chapter{Proposed algorithms}
There are two algorithms: CBST algorithm and terrain monitoring algorithm.
The Alg.~\ref{al:CBST} is to generate the CBST (e.g., from Fig.~\ref{fig:method_intro} (a) to Fig.~\ref{fig:method_intro} (b)).
The Alg.~\ref{al:TM_CBST} is to generate the path using GCB algorithm (e.g.,  from Fig.~\ref{fig:method_intro} (b) to Fig.~\ref{fig:method_intro} (c)).


In Alg.~\ref{al:CBST}, the inputs are an undirected graph ($G$), the source ($s$).
The output is the spanning tree.
Line $1$ is to build $Q$ (vertices in $G$).
Lines $2-3$ are initialization{\color{olive}s}.
Line $5$ is to find the distance connecting $v_q$ and $v_k$, where $v_q$ belongs to the current growing tree ($S$), and $v_k$ does not belong to the tree current growing tree.
Line $6$ is to find the path length from the source vertex to $v_q$ along the current growing tree ($S$).
Line $7$ is to maximize $f/c$, where $c$ is the cost function about Prim's algorithm and Dijkstra algorithm.
Lines $8-9$ are to update current tree and spanning tree.


\begin{algorithm}[htbp]	
	\KwIn{\\
$G = (V, E, w)$: undirected graph \\
$s$: the start point \\
$f$: objective function
}
\KwOut{$S$: spanning tree}
\Parameter{$\alpha$ (between $0$ and $1$)}
	\caption{Cost-benefit spanning tree}
	\begin{algorithmic}[1]
    \State {$Q = V$ \#all of vertices in $G$}
    \State {$Q = Q \setminus \{s\}$}
    \State {$S = \phi$}
    \WHILE {$Q \neq \phi$}
        \State {let $d_{qk}$ be the cost edge \; such that $q \in Q$ and $k \in V \setminus Q$}
        \State {$pl_q$ is distance from source to $v_q$}
        \State {maximize ($\frac{f}{\alpha\times pl_i+d_{iu}}$) where $i\in Q,$ $u\in V\setminus Q$}
        \State {$Q = Q \setminus \{i\}$}
%        \FOR {$u$ which are neighbors of $i$ still in $Q$}
%            \State {$RELAX(u, i, w)$}
            \State {$S = S \cup (u, i)$}
%        \ENDFOR
    \ENDWHILE	
    \end{algorithmic}	
	\label{al:CBST}
\end{algorithm}

\begin{algorithm}[thbp]
\KwIn{\\
$S=(V, E, w)$: spanning tree \\
$s$: the start point \\
$f$: objective function \\
$c$: cost function from spanning tree $S$ \\
$B$: budget
}
\KwOut{$\pi$: subgoal set}
    \caption{Terrain monitoring using CBST}
    \begin{algorithmic}[1]
    \State {$G:=\phi$}
    \State {$\pi:=\phi$}
    \State {$V^{'} = V$}
    \WHILE {$V^{'}\neq \phi$}
        \FOR {$X \in V$}
            \State {$\Delta_f^X:=f(G\cup X)-f(G)$}
            \State {$\Delta_c^X:=c(G\cup X)-c(G)$}
        \ENDFOR
        \State {$X^* = argmax\{\frac{\Delta_f^X}{\Delta_c^X}\}$}
        \IF {$c(G\cup X^*)\le B$}
            \State {$\pi = \pi\cup X^*$}
        \ENDIF
        \State {$V^{'}=V^{'}\setminus X^*$}
    \ENDWHILE
    \end{algorithmic}
    \label{al:TM_CBST}
\end{algorithm}

In Alg.~\ref{al:TM_CBST}, the inputs, $S$, $s$, $f$, $c$, and $B,$ represent spanning tree built from Alg.~\ref{al:CBST}, the start point, objective function, cost function from spanning tree $S$ using shortest path tree, and a cost budget, respectively.
The output is $\pi$ which is the subgoal set with budget constraint.
Lines $1-3$ are initializations.
Lines $4-9$ are to find maximum cost-benefit point in the spanning tree.
Lines $10-12$ are to pick the point subject to budget constraint.
Line $13$ is to avoid that the agent pick{\color{olive}s} the point repeatedly.